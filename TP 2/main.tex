\documentclass{article}
\usepackage{graphicx} % Required for inserting images

\title{Analyse numérique TP2}
\author{Thomas Barrais et Nabil Bargach}
\date{March 2023}

\begin{document}

\maketitle

\section{Exercice 2 : Problème aux limites} 

\subsection{Question 1}
Pour montrer que $u = \sin(2\pi x)$ est un vecteur propre de l'opérateur $\hat{A} = -\frac{d^2}{dx^2} + 2$, nous devons vérifier si l'équation suivante est satisfaite:

$$\hat{A}u = \lambda u$$

où $\lambda$ est la valeur propre correspondante.

En appliquant l'opérateur $\hat{A}$ sur $u = \sin(2\pi x)$, nous obtenons:

$$\hat{A}u = -\frac{d^2}{dx^2}(\sin(2\pi x)) + 2\sin(2\pi x)$$

En utilisant la formule de dérivation de la fonction sinusoïdale, nous avons:

$$\hat{A}u = -(2\pi)^2\sin(2\pi x) + 2\sin(2\pi x) = (2-4\pi^2)\sin(2\pi x)$$

Ainsi, si nous posons $\lambda = 2 - 4\pi^2$, nous avons:

$$\hat{A}u = \lambda u$$

Ce qui montre que $u = \sin(2\pi x)$ est bien un vecteur propre de $\hat{A}$.

Maintenant, pour déduire une solution exacte du problème aux limites précédent, nous devons résoudre l'équation différentielle:

$$-\frac{d^2u}{dx^2} + 2u = \lambda u$$

où $\lambda = 2 - 4\pi^2$ et les conditions aux limites sont définies par:

$$u(0) = u(1) = 0$$

Pour résoudre cette équation, nous posons $u(x) = A\sin(\sqrt{\lambda} x) + B\cos(\sqrt{\lambda} x)$, où $A$ et $B$ sont des constantes à déterminer en utilisant les conditions aux limites.

La première condition aux limites nous donne:

$$u(0) = 0 = A\sin(0) + B\cos(0) = B$$

Ainsi, nous avons $B = 0$ et notre solution devient:

$$u(x) = A\sin(\sqrt{\lambda} x)$$

Ainsi, la solution exacte au problème aux limites est donnée par:

$$u(x) = \frac{\sin(2\pi x)}{\lambda}$$

\subsection{Question 2}
Considérons le problème suivant:

$$-\frac{d^2u}{dx^2} + 2u = f(x)$$

avec les conditions aux limites:

$$u(0) = u(1) = 0$$

Nous allons utiliser la méthode aux différences finies pour discrétiser l'équation différentielle sur une grille de points équidistants $x_i = i\Delta x$, où $\Delta x = \frac{1}{N+1}$ est l'espacement de la grille et $N+1$ est le nombre de points de
la grille. \newline
\newline
Nous approchons la deuxième dérivée de $u$ en utilisant la différence centrée:
La méthode aux différences finies centrées d'ordre 2 pour la dérivée seconde est
obtenue à partir des formules de Taylor en développant $u(x_i + \Delta x)$ et
$u(x_i - \Delta x)$ autour de $u(x_i)$:

$$u(x_i+\Delta x) = u(x_i) + \Delta x\frac{\partial u}{\partial x}(x_i) + \frac{\Delta x^2}{2}\frac{\partial^2 u}{\partial x^2}(x_i) + \frac{\Delta x^3}{6}\frac{\partial^3 u}{\partial x^3} + \frac{\Delta x^4}{24}\frac{\partial^4 u}{\partial x^4}$$

$$u(x_i-\Delta x) = u(x_i) - \Delta x\frac{\partial u}{\partial x}(x_i) + \frac{\Delta x^2}{2}\frac{\partial^2 u}{\partial x^2}(x_i) - \frac{\Delta x^3}{6}\frac{\partial^3 u}{\partial x^3} + \frac{\Delta x^4}{24}\frac{\partial^4 u}{\partial x^4}$$

En ajoutant ces deux expressions et en résolvant pour $\frac{\partial^2 u}{\partial x^2}(x_i)$, on obtient la formule de différences finies centrées d'ordre 2 pour la dérivée seconde:

$$\frac{\partial^2 u}{\partial x^2}(x_i) = \frac{u(x_i+\Delta x) - 2u(x_i) + u(x_i-\Delta x)}{\Delta x^2} + \frac{\Delta x^2}{12}\frac{\partial^4 u}{\partial x^4}$$

$$\frac{d^2u}{dx^2} \approx \frac{u_{i+1}-2u_i+u_{i-1}}{(\Delta x)^2}$$

où $u_i$ est la valeur approchée de $u(x_i)$ sur la grille. En utilisant cette approximation, nous pouvons discrétiser l'équation différentielle:

$$-\frac{u_{i+1}-2u_i+u_{i-1}}{(\Delta x)^2} + 2u_i = f_i$$

où $f_i = f(x_i)$ est la valeur approchée de la fonction $f$ sur la grille. On fixe u(0) = u(N+1) = 0 car $u_{i-1}$ ou $u_{i+1}$ n'existent pas pour ces points.

\subsection{Question 3}

Cette expression peut être écrite sous forme matricielle:

$$AU = b$$

où $U$ est un vecteur colonne contenant les valeurs de $u$ sur les points de la grille, $b$ est un vecteur colonne contenant les valeurs de $f$ sur les points de la grille, et la matrice $A$ est donnée par:

\begin{enumerate}
    \item Noeud 1 : $u_{0}=0$
    \item Noeud 2 : $\frac{u_3 - 2u_2 + u_1}{h^2} + 2u_2 = f_2$
    \item ...
    \item Noeud N : $\frac{u_{N+1} - 2u_N + u_{N-1}}{h^2}=f_N$
    \item Noeud N+1 : $u_{N+1}=0$
\end{enumerate}

Le second membre de ces expressions forme la matrice colonne b. Les lignes de la  matrice A sont formées des coefficients devant les $u_i$.

\section{Exercice 3 : Equation de la chaleur}
\subsection{Question 1}

Pour montrer que le schéma précédent s'écrit vectoriellement, nous allons utiliser la notation matricielle pour représenter les inconnues $u_{j,i}$.

Soit $U^j = (u_{1}^j, u_{2}^j, \cdots, u_{n-1}^j, u_{n}^j)$ un vecteur colonne de taille $n-1$. 
\newline On pose également $V^j = (\alpha^j, 0, \cdots, 0, \beta^j$ un vecteur colonne de taille $n-1$.
\newline
\newline
Equation du schéma explicite :

$$\frac{u_i^{j+1} - u_i^{j}}{\Delta t} - \frac{u_{i+1}^j - 2u_i^j + u_{i-1}^j}{h^2} = 0$$

Ainsi, on peut écrire ces équations sous forme matricielle :

$$\frac{U^{j+1} - U^{j}}{\Delta t} + A_h U^j = \frac{1}{h^2}V^j$$
\newline
\newline

Avec $A_h$ de la forme :
\begin{itemize}
    \item $\frac{2}{h^2}$ sur la diagonale 
    \item $-\frac{1}{h^2}$ sur les diagonales supérieur et inférieur
\end{itemize}

\subsection{Question 2}
En modifiant l'équation précédante, avec $A_h = \frac{A}{h^2}$ :

$$U^{j+1} - U^j + \Delta t A_h U^j = \frac{\Delta t}{h^2}V^j$$

Ce qui nous donne :

$$ U^{j+1} + [-I_n + \frac{\Delta t}{h^2}A]U^j = \frac{\Delta t}{h^2}V^j$$

Avec $I_n$ la matrice identité d'ordre n
\newline

D'après les données de l'énoncé on obtient une matrice A 3x3, h = 0.25 et $\Delta t = 0.001$
\newline
\newline
On sait que : $U^0 = \frac{f(x_1)f(x_2)}{f(x_3)}$, or $\forall i f(x_i) = 50$, d'où:

$$(U^0)^T = (50 \ 50 \ 50)$$

De plus, $u_0^j = \alpha$ et $u_2^j = \beta$, on obtient donc:

$$(V^j)^T = (\alpha \ 0 \ \beta)$$

La matrice A s'obtient facilement avec les informations sur $A_h$ données à la question 1.

\subsection{Question 5}

En utilisant comme approximation de la dérivée simple : $\frac{\partial u}{\partial t}(x_i, t_j) \simeq \frac{u_i^j-u_i^{j-1}}{\Delta t}$, on obtient comme schéma explicite du problème :

$$U^j - U^{j-1} + \Delta t A_h U^j = \frac{\Delta t}{h^2}V^j$$

Ce qui nous donne :

$$U^{j-1} = [\frac{\Delta t}{h^2}A + I_n]U^j - \frac{\Delta t}{h^2}V^j$$

\subsection{Question 6}

La formulation explicite permet de modéliser plus finement les phénomènes mais est plus gourmande en ressources. La méthode implicite quant à elle est plus stable


\end{document}
